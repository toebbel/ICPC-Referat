\documentclass[hyperref={pdfpagelabels=false}]{beamer}
\usepackage{lmodern}

\usepackage[utf8]{inputenc} % this is needed for german umlauts
\usepackage[ngerman]{babel} % this is needed for german umlauts
\usepackage[T1]{fontenc}    % this is needed for correct output of umlauts in pdf

\usepackage{verbatim}
\usepackage{tikz}
\usetikzlibrary{arrows,shapes}

% Define some styles for graphs
\tikzstyle{vertex}=[circle,fill=black!25,minimum size=20pt,inner sep=0pt]
\tikzstyle{selected vertex} = [vertex, fill=red!24]
\tikzstyle{edge} = [draw,thick,-]
\tikzstyle{weight} = [font=\small]
\tikzstyle{selected edge} = [draw,line width=5pt,-,red!50]
\tikzstyle{ignored edge} = [draw,line width=5pt,-,black!20]

\usetheme{Frankfurt} % see http://deic.uab.es/~iblanes/beamer_gallery/index_by_theme.html
\usefonttheme{professionalfonts}
\beamertemplatenavigationsymbolsempty

\begin{document}
\title{Graphentheorie II}   
\author{Martin Thoma, Tobias Sturm, Max Wagner, Thomas Krings} 
\date{\today} 
\subject{Graphentheorie-Referat fur ICPC}

\frame{\titlepage} 

\frame{
	\frametitle{Inhaltsverzeichnis}
	\tableofcontents[section]
}

\input{MinimaleSpannbaume}       % Minimale Spannbäume
\input{SCC}       % Starke zusammenhangskomponenten
\input{GraphColoring}  % Färbung von Graphen
\input{Kreise}  % Färbung von Graphen

\end{document}